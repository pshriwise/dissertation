
\chapter{Introduction}\label{ch:introduction}


Methods for modeling radiation transport determine particle flux, or derived
quantities, across space, angle, energy and time. The combination of the space,
angle, energy, and time domains is known as \textit{phase space}. The behavior
of these particles is described by the linear Boltzmann transport
equation \cite{Ulam_1949}. Deterministic solve this transport equation by discrectizing
the phase space of the problem, but time and memory constraints often limit the
resolution of phase space in practical problems.

The Monte Carlo approach to modeling Radiation transport simulates the
interaction of individual particles across the phase
space \cite{Lewis_1993}. This method was developed at Los Alamos National
Laboratory (LANL) during World War II by Fermi, von Neumann, Ulam, Metropolis,
and Richtmyer \cite{LANL_1987}. It uses a random walk process to solve the
transport equation. Pseudo-random numbers are used to sample probability
distribution functions representing properties of the virtual medium and in turn
determine the particle interaction outcomes. This stochastic approach requires
the simulation of many particles to reduce the statistical uncertainty of the
solution, where the uncertainty is inversely proportional to the square root of
the number of particles simulated. As the number of simulated particles
approaches infinity, tallied quantities approach the value of the continuous
solution.

The pros and cons of the two approaches complement each other. While
deterministic approaches inherently calculate a solution over the entire problem
domain, they take on additional error by discretizing phase space. In contrast,
Monte Carlo methods only incur error associated with input parameters such as
cross sections or geometry specifications, but it is challenging to achieve a
global solution with constant statistical error using this
approach. Computationally, deterministic methods typically suffer memory and
runtime costs that scale with the resolution of the discretized phase
space. Monte Carlo methods are typically limited by the runtime needed to
achieve satisfactory uncertainty in a region of interest.


\section{Monte Carlo Geometry}

Historically, Monte Carlo codes Constructive Solid Geometry (CSG) as their
\textit{native} geometry representation. CSG representations represent 3D
regions of virtual space using Boolean combinations of half spaces defined by
quadratic surfaces.To define the geometry, the surface definitions and the
Boolean combinations used to represent 3D regions (also referred to as cells or
volumes) are entered into a text file. This format for geometry is robust once
defined properly, but is limited in representation compared to more modern
geometric modeling tools such like Computer-Aided Design (CAD).

CAD allows for increased accuracy in model representation and better human
efficiency. CAD is able to represent higher-order surfaces and provides access
to models used for analysis in other engineering domains. These shared models
allow for a common problem domain in coupled simulation. CAD systems also provides a
rich set of tools for model generation, topological representation, and design
iteration. In highly complex and well-developed models, all of these tools are
more intuitive and efficient for human use over alteration of native text-based
formats. Many tools exist for converting native CSG models to and from CAD
systems, and a few have the capability to perform simulations directly on CAD
geometries.


\section{Monte Carlo Geometry Queries}

