

\newcommand{\precondQuery}[4] {
  \null %emptyline
  \textbf{\uppercase{#1}} 
  \begin{adjustwidth}{1em}{0pt}
    \begin{figure}[H]
      \begin{center}
        \includesvg{../images/#2}{0.65\textwidth}
        \caption{#3}
        \label{fig:#2}
      \end{center}
    \end{figure}
    #4
  \end{adjustwidth}
}

\chapter{Signed Distance Field Preconditioning}\label{ch:preconditioning}

This chapter describes the adaptation of a rendering data structure, the signed
distance field, as a tool for accelerating CAD-based transport in the Direct
Accelerated Monte Carlo (DAGMC) toolkit. A model for predicting the data
structure's utilization is also introduced. Finally, demonstrations of its
effectiveness for a number of simple problems and production models are shown
and discussed.

\begin{figure}[H]
  \includesvg{../images/preconditioner_datastruct}{0.5\textwidth}
  \centering
  \caption{2D visualization of the signed distance field with sign conventions
    reversed for use in radiation transport.}
  \label{fig:preconditioner_datastruct}
\end{figure}

\section{Preconditioning Theory}\label{section:preconditioner_theory}

Of the geometric queries that DAGMC supports, next surface intersection, point
containment, and closest to location are most commonly called in simulations.
Typically, a ray is fired to satisfy any of these queries in DAGMC with
$O(logN)$ complexity using MOAB's BVH, but it is hypothesized that these queries
can be accelerated in many cases by first performing an $O(1)$ signed distance
value look-up to precondition ray fire calls.


\precondQuery{Point Containment}
              {point_containment_sdf}
              {Examples of scenarios for point containment preconditioning using signed distance values.}
              {
                Point containment queries can be preconditioning by examining
                the interploated signed distance value for the current particle
                location. If the point's value is negative (or outside of the
                SDF), then the point is considered to be outside of the
                volume. If the point's value is positive, then the point is
                determined to be inside the volume.

                Given that there is error associated with each of these
                interpolated values, the result of this method should only be
                trusted if the absolute value of the signed distance is greater
                than the expected error associated with the value. The error If
                this is not the case, then a ray must be fired to determine the
                particle's containment with respect to the volume in
                question. In effect, this verifies that the location is far
                enough from the boundary of the volume to make a definititive
                statement about whether it is inside or outside of the volume
                based on the sign of it's interpolated signed distance value.
                Figure \ref{fig:point_containment_sdf} graphically describes the
                outcome of these different cases in 2D.
              }

\precondQuery{Next Surface}
             {next_surface_sdf}
             {Examples of scenarios for next surface intersection preconditioning using signed distance values.}
             {
               Next surface intersection queries are the most common geometry
               query in Monte Carlo simulations. The queries are intiated by
               native Monte Carlo codes to determine if a particle will cross a
               surface before reaching its next physics event
               location. Normally in DAGMC a ray is fired each time this query
               is called. This can be avoided by using the signed distance
               field to exclude the possibility of a surface crossing without
               explicitly determining the next surface intersection. If the sum
               of the signed distance values for both the current particle
               position and the next physics event location is greater than the
               distance between the two, then no surface crossing will occur
               and the particle can safely advance to the next physics event
               location. Figure \ref{fig:next_surface_sdf} depicts these
               different cases in 2D space.
               
               For robustness, the error for each interpolation should be
               subtracted from the sum of the signed distance values as a
               protective measure against invalid surface crossings. If the
               expanse between the particle's current location and its next
               physics event location cannot be accounted for by the signed
               distance values of the two points, then the next surface
               intersection will be found using a ray fire
               call. Is it acknowledged that not all
               Monte Carlo codes provide the next physics event location along
               with the particle's current location to their geometry
               kernels. In this case, preconditioning of these queries in this
               manner will not be possible.
              }

\precondQuery{Closest Surface}
             {closest_surface_sdf}
             {Examples of scenarios for closest surface preconditioning using signed distance values.}
             {
               Closest surface queries can be performed in a similar manner to
               the point containment queries, but they are more dependent on the
               native code's intent for their use. Some Monte Carlo codes always
               query for the closest surface intersection in order to
               determine whether or not the particle will exit the volume before
               reaching its next physics event location. This information can be
               interpolated from a signed distance field to the same effect.
               
               In similar fashion to the point containment case, the signed
               distance value should only be trusted if it is greater than the
               error associated with the value. Additionally, the error should
               be subtracted from the value, returning to the code a
               conservative value for the nearest intersection. If the signed
               distance value's magnitude is not greater than its error
               evaluation or if the value is negative, then a ray should be
               fired to determine the exact location of the nearest boundary
               crossing for the particle's location. Figure
               \ref{fig:closest_surface_sdf} depicts these different cases in a
               2D example.
             }

%% \begin{figure}[ht]
%%   \center
%%   \includesvg{../images/preconditioner_ex}{0.35\textwidth}
%%   \caption{Visualization of ray preconditioning scenarios. The $\epsilon$ here represents
%%     the associated error of the signed distance value interpolation.}
%%   \label{fig:precondition_ray}
%% \end{figure}

\bigskip
             
Using these methods, signed distance fields can act as a preconditioning tool
for the relatively expensive ray fire process to accelerated CAD-based MCRT by
using an $O(1)$ process to establish that the conditions of a geometric query
are such that a more computationally expensive ray fire is necessary before
performing the ray tracing operation. It is hypothesized that for some Monte
Carlo problems this process can be used to avoid $O(logN)$ ray fire calls to
significantly reduce the runtime of the simulation.

\section{Implementation in MOAB}

\subsection{SDF Construction}

MOAB provides an interface for construction of structured mesh which stores
explicit vertex coordinates, hex elements, and entity handles. As with any
entity stored in MOAB, tag data can be applied to these elements. These vertex
coordinates and hex elements can be accessed using $<i,j,k>$ indexing, where the
coordinate $<0,0,0>$ and  $<N_{x}, N_{y}, N_{z}>$ represent the lowest and
highest corner of the structured mesh in
parameter space for a mesh containing $N_{x},N_{y},N_{z}$ elements in the x, y,
and z directions respectively. This representation provides
a fast path for verification, visualization, and proof of concept in transport
test cases for demonstration, but is very memory intensive. An implicit version
of the structured mesh has been implemented which takes advantage of the uniform
step size in each dimension to store only: the number of elements in the mesh,
the location of the lowest corner in that mesh, and a flat array of the signed
distance values associated with the elements of the mesh.
%A much less memory intensive implementation in which only a box
%corner, grid step size, dimensions, and the signed distance value data are
%stored has now also been implemented.
This avoids the storage of vertex coordinates, mesh element connectivity, and
all associated handles to those entities. There is an added cost in re-computing
vertex coordinates when a signed distance value is interpolated, but this added
cost is negligible in comparison to a ray traverval and memory is of greater
concern for this spatially dense data structure. This data structure can still
interact with MOAB's structured mesh interface to generate an explicit mesh for
visualization and verification if required.


%% An initial implementation of the signed distance field employed MOAB's
%% structured mesh interface and data tagging capabilites for storage of the data
%% structure. This interface maintains a representation of the structured mesh with
%% vertex coordinates and handles at each point along with hex elements for each
%% mesh voxel. While this format is somewhat memory intensive, it provides
%% a fast path for verification, visualization, and proof of concept in transport
%% test cases for demonstration.

\begin{figure}
  \begin{center}
  \includegraphics[scale=0.35]{../images/sdf_sphere.png}
  \caption{A visual of a signed distance field with step size 0.5 cm surrounding
    the spherical volume of test case with a radius of 10 cm. Note the reversal
    of the sign convention in comparison to Eq. \ref{eq:sdf}. It is preferable
    to change this when populating the data structure rather than incurring the
    additional computational cost of altering the sign of values for each
    operation at runtime.}
  \end{center}
  \label{fig:sdf_sphere}
\end{figure}

Signed distance values can be retrieved from the structured mesh by determining
which mesh voxel the point lies within. The point's element is accessed by
determining an $<i,j,k>$ index using the point's x, y, and z values divided by
the structured mesh step size. A trilinear interpolation of the mesh element's 8
vertex coordinates and their signed distance values is then used to provide the
signed distance value for the location of interest. As a result, the complexity
of a signed distance value lookup from the signed distance field is
$O(1)$.

As an initial implementation, one signed distance field is generated for each
volume in DAGMC with extents matching the axis-aligned bounding box of the
volume. The signed distance field is represented as a uniform structured mesh
with a signed distance value at each vertex in the mesh as indicated in Fig.
\ref{fig:preconditioner_datastruct}.

\subsection{SDF Population}

Signed distance fields are typically generated using an implicit, analytic
representation, but a suitable data structure for populating the structured mesh
with signed distance values is already in place in the form of DAGMC's bounding
volume hierarchy. It is a more straightforward process to simply use DAGMC's
current closest to location algorithm to generate signed distance values than to
create an implicit surface approximation of the triangle mesh. This method also
maintains a consistency between the intersections found by the ray tracing
kernel and the signed distance field values.

DAGMC's closest to location algorithm returns, among other pieces of
information, the nearest intersection location and the triangle on which this
intersection exists. For each point in the signed distance field mesh, this
algorithm is used to determine the magnitude of the distance value. To
accomplish this, a ray is constructed from query location to the intersection
location. The dot product of this ray vector with the triangle's outward normal
vector is used to determine the sign of the distance value. DAGMC maintains
enough information to consistently orient triangle normals such that they point
outward from the volume they represent. In the rare cases for which the dot
product of these vectors is ambiguous, or zero, DAGMC's point containment
algorithm is used to disambiguate the value's sign.

\section{Utilization Modeling}\label{section:preconditioner_utilization}

In effect, the preconditioner is attempting to check whether or not the particle
will actually cross a surface before explicitly searching for the particle's
intersection with a surface along its current trajectory. If the result of this
preconditioning check is always false and a ray is always fired, then these
checks are only adding to the computational cost of the problem. This will
always occur in volumes filled with void, for example, as particles immediately
travel from one side of a volume to another. As a result, the signed distance field
should be applied selectively depending on each volume's geometric and material
properties for optimal performance and high utilization of the preconditioning
methods.

Ideally, this method will only be applied to volumes in which the data
structure is able to precondition ray fire calls often or with high utilization
of the data structure. The signed distance field is expected to have the biggest
impact in performance when preconditioning next surface intersection queries, as
they are most commonly called in Monte Carlo codes when tracking neutral particles
through the geometry. As such, this type of query is the focus of utilization
measurement for the remainder of this section.

PROVIDE PROOF THAT NEXT SURFACE IS MOST CALLED

\begin{equation}
  U = \frac{ \small \text{Rays Avoided w/ SDF} }{ \small \text{Number of Geometry Queries} } 
   \label{eq:preconditioner_utilization}
%  \caption{Definition of signed distance field utilization as a ray fire preconditioner in DAGMC.}
\end{equation}

As shown in Eq. \ref{eq:preconditioner_utilization}, the utilization, $U$, of
the signed distance field as a ray fire preconditioner can be described as the
number of ray fire calls related to the next surface intersection queries that
are avoided divided by the total number of next surface intersection queries
made by the Monte Carlo code. This value can be quantified using this definition
using debugging tools, such as Valgrind, to determine the number of queries made
in DAGMC and the number of rays fired inside of the subroutine.  It is expected
that in the majority of cases, as the utilization of preconditioning methods
goes up, the performance of the simulation will also improve.



\begin{figure}[ht]
  \centering
  \includesvg{../images/sdf_hydrogen_density_study_util}{\textwidth}
  \caption{Utilization results for a 5 MeV neutron source at the origin of a 10 cm radius
    sphere. Hydrogen density was varied from 0 to 1 $\frac{g}{cm^3}$.}
  \label{fig:sphere_hydrogen_density_study_util}
\end{figure}

To understand this utilization more deeply with respect to material parameters,
the hydrogen density was varied from 0 to 1 $\frac{g}{cm^3}$ in the single-volume
sphere test problem with a 5 MeV neutron point source. For each density, one
simulation was performed without the signed distance field and another with the
signed distance field and preconditioning enabled. Fig.
\ref{fig:sphere_hydrogen_density_study_util} shows the utiilzation results of this study. 

\begin{figure}[ht]
  \centering
  \includesvg{../images/sdf_hydrogen_density_study_perf}{\textwidth}
  \caption{Performance results for a 5 MeV neutron source at the origin of a 10
    cm radius sphere. Hydrogen density was varied from 0.0 to 1.0
    $\frac{g}{cm^3}$. Simulations of 100M histories at each density were
    performed using native MCNP5, DAG-MCNP5 without the signed distance field,
    and DAG-MCNP5 with the signed distance field.}
  \label{fig:sphere_hydrogen_density_study_perf}
\end{figure}

Utilization of the data structure in this study remains high until the hydrogen
density falls to 0.1 $\frac{g}{cm^3}$ at which point a distinct knee appears and
the utilization falls off quickly. Even at the lowest density reached in the
study of 0.01 $\frac{g}{cm^3}$, the utilization of the signed distance field to
avoid ray fire calls is 0.54.


Fig. \ref{fig:sphere_hydrogen_density_study_perf} provides an impression of the
performance of these three implementations converge as the density of the
hydrogen is varied. As the hydrogen density approaches 1.0 $\frac{g}{cm^3}$, a
factor of ~3.5 improvement in runtime is seen in the simulation where the SDF is
applied as a preconditioner. The application of the signed distance field allows
for significantly improved performance until the density drops below 0.1
$\frac{g}{cm^3}$ in agreement with utilization plot. As the material density
decreases, particles quickly leave the geometry after very few collisions. It is
difficult to judge the impact on the performance of this simulation for these
low density values due to the limited size of the geometry and the short lived
histories.  In order to have more control over a simulation's physical
parameters, subsequent experiments were performed using a pseudo Monte Carlo
simulation tool.

\section{Signed Distance Field Utilization Modeling}

In order to characterize utilization of a signed distance field as a
preconditioner for next surface intersection queries in DAGMC, a pseudo Monte
Carlo simulation tool was developed using DAGMC. This tool was used to simulate
different transport scenarios within a spherical geometry using an isotropic
volumetric source and isotropic scattering. Particle histories are terminated
based on a maximum number of collisions or departure from the problem
geometry. Particle distance traveled, $d$, can be represented by either a fixed
distance or by sampling for the standard probability of interaction in a medium
with mean free path, $\lambda$. The tool allows the value of $\lambda$ to be set
directly, enabling a relation between the signed distance field and this value
to be developed with intent for use this relationship as a means for
characterizing appropriate conditions for application of the signed distance
field.

\begin{figure}[!htb]
  \centering
  \includesvg{../images/sdf_fixed_dist_results}{\textwidth}
  \caption{Results of the model for the theoretical utilization limit with the
    results of the simulation for a fixed distance traveled case.}
  \label{fig:sdf_fixed_dist}    
\end{figure}

To begin, simulations were performed for particles with a fixed distance
traveled for varying distances and signed distance field step sizes. Run times
of the simulation are not shown here as the data structure's utilization is the
main focus of this study. The results of the study are shown in
Fig. \ref{fig:sdf_fixed_dist}. As the signed distance field mesh step size
increases, utilization of the data structure decreases due to the increasing
error associated with the interpolation of signed distance values. Additionally,
utilization is expected to decrease with increasing distance traveled. This
decreased utilization is caused by not only the increased distance between the
two particles, but also by the increased probability that both locations will be
closer to surfaces of the sphere and have smaller signed distance values. A
theoretical limit for the utilization is also shown in
Fig. \ref{fig:sdf_fixed_dist}. The development of the analytic form for this
limit will now be discussed.

\begin{figure}[ht]
  \centering
  \includesvg{../images/alpha}{0.3\textwidth}
  \caption{Depiction of model variables.}
  \label{fig:model}
\end{figure}

The utilization of the signed distance field as a preconditioner for ray tracing
operations can be modeled as an evaluation of the combined probability space for
particles with a current position, $\vec{p}$, and a next physics event location,
$\vec{n}$, after traveling a distance, $d$. The fraction of this probability
space in which signed distance values can be used to rule out surface crossings
for next surface intersections is then considered to be the theoretical
utilization of the signed distance field. An initial form for this probability
space can found in Eq. \ref{eq:util_model}.

\begin{equation}
  \label{eq:util_model}
\int_{V_{sphere}}\int_{V_{track}} p_p(r) p_n(d) \, \mathrm{d}V_{sphere}\mathrm{d}V_{track}
\end{equation}

In this model, the starting location of particles, $\vec{p}(r,\phi,\theta)$, is
uniformly distributed, $p_p(r)=1$, throughout a sphere of radius, $R$.  The
location of the next event, $n(d,\alpha,\beta)$, where $d$ is the distance
traveled by the particle, $\alpha$ is the interior angle between the
particle's \textit{position} vector and the particle's sampled direction
vector, and $\beta$ represents an azimuthal angle for directions traveled with
angle of departure, $\alpha$. Fig. \ref{fig:model} depicts these
variables, $r$, $d$, and $\alpha$ more clearly.

The outer integral in Eq. \ref{eq:util_model} represents all possible particle positions within the
geometric sphere and expands to

\begin{equation}
\int_{0}^{R}\int_{0}^{2\pi}\int_{0}^{\pi}\int_{V_{track}} r^2\sin{\phi} \, \mathrm{d}\phi
\mathrm{d}\theta \mathrm{d}r \,  p_n(d) \mathrm{d}V_{track}
\end{equation}

The inner integral over $V_{track}$ then expands to

\begin{equation}
\small \int_{0}^{R}\int_{0}^{2\pi}\int_{0}^{\pi}\int_{0}^{\infty}\int_{0}^{2\pi}\int_{0}^{\pi}
r^2\sin{\phi} \, p_n(d) d^2 \sin{\alpha} \, \mathrm{d}\alpha \mathrm{d}\beta \mathrm{d}d \, \mathrm{d}\phi
\mathrm{d}\theta \mathrm{d}r
\end{equation}

Integration of $\phi$, $\theta$, and $\beta$ can now be performed with
the knowledge that they are symmetric with respect to the problem and
integration of $p_n(d)$ does not rely on them.

\begin{equation}
\small 8\pi^2  \int_{0}^{R}\int_{0}^{\infty}\int_{0}^{\pi} p_n(d) \,
r^2 \, d^2 \sin{\alpha} \, \mathrm{d}\alpha \mathrm{d}d \, \mathrm{d}r
\end{equation}

In order to represent particles traveling a fixed distance, the relationship in Eq. \ref{eq:pn_fixed}
is applied.

\begin{equation}
  \label{eq:pn_fixed}
  p_n(d) = \frac{\delta(d-\lambda)}{d^{2}}
\end{equation}

The evaluation of this integral then gives a representation of all the query
space available to the problem

\begin{equation}
  \label{eq:A_fixed}
\small A = 8\pi^2  \int_{0}^{R}\int_{0}^{\infty}\int_{0}^{\pi} \delta(d-\lambda) \,
r^2 \, \sin{\alpha} \, \mathrm{d}\alpha \mathrm{d}d \, \mathrm{d}r
\end{equation}

and represents all geometric query space, labeled $A$, for a sphere of radius,
$R$ and a fixed distance traveled, $\lambda$.

In order to understand what fraction of this query space is able to be
preconditioned, the condition for avoiding an explicit nearest intersection
search along a particle direction in Eq. \ref{eq:condition} will now be
applied.

\begin{equation}
  SDV(\vec{p}) + SDV(\vec{n}) > |\vec{p}-\vec{n}| + 2\varepsilon(h)
  \label{eq:condition}
\end{equation}
\begin{align*}
 &SDV - \, signed \, distance \, value \, function \\
 &\vec{p} - \, particle's \, current \, position \\
 &\vec{n} - \, particle's \, next \, event \, location \\
 &h - \, mesh \, step \, size \\
 &\varepsilon(h) - \, error \, evaluation \, for \, signed \, distance \, values \\
\end{align*}

This condition establishes that the nearest location to intersection for both
points must be greater than the distance between the two points plus any error
associated with their signed distance values as previously discussed. This
condition is true for some fraction of the next surface queries in a Monte Carlo
simulation, but not all.  This model does not account for error, making it an
idealized model representing the largest utilization possible for any given
pseudo simulation.

\begin{equation}
SDV(\vec{x}) =  R-|\vec{x}|
\end{equation}

Making these substitutions into the inequality gives

\begin{equation}
R-|\vec{p}| + R - |\vec{n}| >   |\vec{p}-\vec{n}|
\end{equation}
The right hand side of this inequality can be described as the distance
traveled, $d$, and the magnitude of $\vec{p}$ can be represented
by the variable $r$.

\begin{equation}
 R-r + R - |n(d,\alpha,\beta)| > d
\end{equation}

Reducing the next event location, $\vec{n}(d,\alpha,\beta)$, into an expression
in terms of $r$, $d$, and $\alpha$ requires further examination of the
problem. Because the coordinates of $n$ depend on the current particle position,
the magnitude of $n$ with respect to the geometry origin must be obtained to get
a correct form for the signed distance value. Again, Fig. \ref{fig:model} depicts the
value of $n$ graphically for reference. The magnitude of n can then be described
using the law of cosines as

\begin{equation}
|n(d,\alpha,\beta)| = \sqrt{r^2 + d^2 - 2rd \cos{\pi-\alpha}}
\end{equation}
inserting this into the inequality gives

\begin{equation}
R-r + R - \sqrt{r^2 + d^2 + 2rd \cos{\alpha}} > d
\end{equation}

The inequality has now been reduced to the three variables seen in
Eq. \ref{eq:A_fixed}:$r$, $d$, and $\alpha$. This inequality can be applied to
construct limits of integration representing boundaries of space in which the
SDF can be utilized. By rearranging the inequality, a limit on the angle of
departure, $\alpha$, from the particle's position can be derived.

\begin{equation}
\alpha_{min} > \arccos\Bigg ( \frac{(2R-r-d)^2-d^2-r^2}{2 d r} \Bigg )
\end{equation}

\begin{figure*}[ht]
  \centering
  \includesvg{../images/model_cases_fixed_distance}{\textwidth}
  \caption{Depiction of modeling cases. Left: an example of a track for which
    $d < R - r$. Middle: an example of a track for which $R-r < d < R$ and can be
    preconditioned.  Right: an example of a track for which $R-r < d < R$ and
    cannot be preconditioned.}
  \label{fig:modeling_cases}
\end{figure*}

This condition on alpha can be interpreted as a minimum interior angle that the
particle's trajectory must take relative to the particle's position vector,
$\vec{p}$, for a distance traveled, $d$, for a ray fire to be avoided and the
preconditioner to be utilized. The examination of this condition as a function
of the distance traveled for various values of $r$ results in some conclusions
about how signed distance values are being utilized.

\begin{figure}
\centering
\includesvg{../images/alpha_r}{\textwidth}
\caption{Plot of minimum angle of departure restriction for particles with
various radial positions, $r$, in a sphere with R = 100 cm and varying distances traveled.}
\end{figure}

The inequality is undefined until the distance reaches a value $d = R- r$. This
is because the angular limit only needs to be applied to areas of the query
space in which the distance traveled is large enough to violate the above
condition as depicted in Fig. \ref{fig:modeling_cases}. A violation of this
limit may only occur when a particle travels far enough to reach the geometric
sphere boundary along the current position vector as if it were moving directly
toward the boundary of the sphere. An additional interesting feature of this
plot is the convergence of all the curves on $\pi$ as $d$ approaches $R$. The
convergence on $\pi$ indicates that as the distance traveled approaches $R$ the
only direction that the particle can move is back toward the origin along the
position vector. It also defines a maximum distance a particle can travel in the
sphere and still be preconditioned using signed distance values. Intuitively
this makes sense as the maximum chord length of a sphere is $2R$, and once a
particle travels a distance $R$ the sum of the signed distance values can then
be no larger than $R$ and the condition for utilization in Eq. \ref{eq:condition} is
violated. Hence all curves go to zero at $\lambda = 100 cm$ in
Fig. \ref{fig:sdf_fixed_dist}.

In order to account for the fact that the form of $\alpha_{min}$ is undefined
until $d = R-r$, a Heaviside function is applied before applying it as a limit
on the particle's angle of departure from the position vector. Similarly,
because the $\alpha_{min}$ condition is undefined after $d=R$ a Heaviside
function is used to limit the condition to $\pi$ for any distances traveled
larger than $R$.

\begin{equation}
  \small
  \begin{split}
  \alpha_{min} =& (H(d-(R-r))-H(d-R)) \arccos\Bigg ( \frac{(2R-r-d)^2-d^2-r^2}{2 d r} \Bigg ) \\
  &+ \pi \, H(d-R)
  \end{split}
\end{equation}


By inserting this condition as a lower limit of the $d\alpha$ integration, the
following integral will give all utilized space, $US$, in the query space of the
simulation.

\begin{equation}
  \label{eq:subs_a_cond}
\small US = 8\pi^2  \int_{0}^{R}\int_{0}^{\infty}\int_{\boldsymbol{\alpha_{min}}}^{\pi} \delta(d-\lambda) \,
r^2 \, d^2 \sin{\alpha} \, \mathrm{d}\alpha \mathrm{d}d \, \mathrm{d}r
\end{equation}

Evaluating this integral and dividing by all query space gives the
following form for the theoretical limit of signed distance field utilization as a
preconditioner for ray firing

\begin{equation}
U_{theoretical} = \frac{US}{A} =  \frac{(1-H(\lambda-R))(2R-\lambda)(R-\lambda)}{2R^2}
\end{equation}

It can be seen in Fig. \ref{fig:sdf_fixed_dist} that this utilization limit
works well as an upper limit for the simulation results using various signed
distance field mesh resolutions. As the step size of the mesh approaches zero,
so does the evaluation of the error, resulting in the same utilization curve
with varying distance traveled, $\lambda$, as in the analytic form developed
here. Future work will include the comparison of this utilization limit to other
single-volume geometries using dimensionless parameters to determine if the
model above can be used to predict signed distance field utilization in other
geometries as well.

%% After the agreement of the simulation results and analytic model for signed
%% distance field utilization for the fixed distance traveled case, the simulation
%% was used to produce a similar set of results in which the distance is
%% sampled based on the standard probability for distance to interaction in a
%% medium with a cross section, $\Sigma$, or mean free path $\lambda
%% =1/\Sigma$. This results in the probability distribution function shown in
%% Eq. \ref{eq:pn_sampled} for the particle distance traveled in this scenario.

%% \begin{equation}
%%   \label{eq:pn_sampled}
%% p_n(d) \propto \frac{e^{-\Sigma d}}{d^{2}} = \frac{e^{-\frac{d}{\lambda}}}{d^{2}}
%% \end{equation}

%% \begin{figure}[ht]
%% \centering
%% \includesvg{../images/sdf_sampled_dist_results}{0.5\textwidth}
%% \caption{Results of the model for the theoretical utilization limit with the
%% results of the simulation for a sampled distance traveled case.}
%% \label{fig:sdf_sampled_dist}
%% \end{figure}

%% %has to be in this section for latex reasons. grumble grumble...
%% \begin{table*}[!h]
%%   \centering
%%   \begin{tabular}{lcccc}
%%           \multicolumn{5}{l}{\textbf{\textit{Source Location:}} <0,0,-1>} \\
%%           \textbf{Implementation} & \textbf{ctme (min)} & \textbf{wall time
%%             (min)} & \textbf{time ratio} & \textbf{precond. utilization}\\
%%           \hline
%%           MCNP6 & 0.17 & 0.14 & 1 & N/A \\
%%           DAG-MCNP6 & 1841.33 & 1841.33 & ~11,000 & N/A \\
%%           DAG-MCNP6 w/ SDF & 0.48 & 0.46 & 2.82 & 0.94\\
%%           \multicolumn{5}{l}{} \\
%%           \multicolumn{5}{l}{\textbf{\textit{Source Location:}} <0,0,10>} \\
%%           \textbf{Implementation} & \textbf{ctme (min)} & \textbf{wall time
%%             (min)} & \textbf{time ratio} & \textbf{precond. utilization}\\
%%           \hline
%%           MCNP6 & 0.18 & 0.18 & 1 & N/A \\
%%           DAG-MCNP6 & 11.12 & 11.16 & 62 & N/A \\
%%           DAG-MCNP6 w/ SDF & 0.50 & 0.52 & 2.89 & 0.96 \\
          
%%   \end{tabular}
%%   \caption{Performance results for an MCNP6 test case involving electron
%%     transport of a 1 keV-100 keV photon source incident on an Fe/W target. 5,000
%%     histories were run in this test problem.}
%%   \label{tab:inp066_results}
%% \end{table*}

%% Following the same process as in the fixed distance case by plugging Eq. \ref{eq:pn_sampled} into
%% Eq. \ref{eq:subs_a_cond}, the utilization form for the sampled distance case is
%% shown in Eq. \ref{eq:sampled_limit}.

%% \begin{equation}
%%   \label{eq:sampled_limit}
%%   U_{theoretical} = \frac{US}{A} = \frac{ \frac{1}{2} \lambda(R - 2 \lambda) e^{\frac{-R}{\lambda}} + \lambda^2 - \frac{3}{2} R \lambda + R^2 }{R^2}
%% \end{equation}

%% The results of this set of simulations can be seen in
%% Fig.\ref{fig:sdf_sampled_dist}. In this scenario, it is not expected that the
%% utilization will approach zero when $\lambda = 100\, cm$, as the actual distance
%% sampled may be considerably less than the provided mean free path for the
%% simulation. Overall utilization values in this scenario for $\lambda$ from 0 to
%% 100 cm remain higher than the corresponding fixed distance simulation cases as
%% is expected in a sampled distance case. Utilization values remain high for
%% relatively large increases in mesh step size, $h$. This is important to
%% application of the data structure given concerns regarding its potentially high
%% memory footprint for large volumes. For example if the utilization of the signed
%% distance field drops $~20\%$ when going from a step size of 1 cm to 6.21 cm, but
%% the memory footprint of the data structure will have decreased by a factor of
%% $6.21^3$ or $239.5$ as well. The optimization of the mesh step size with respect
%% to its effect on utilization will also need to be included in future models of
%% the utilization.

The fixed distance traveled scenario provides a nice baseline for agreement
between the computational results and the model, but a more realistic scenario
is required before attempting to apply the model as a predicitive utility for
application of the SDF in true Monte Carlo codes. The most useful alteration of
the current simulation scenario is to move from a fixed distance distance to a
sampled distance based on the standard probability for distance to interaction
in a medium given a cross section, $\Sigma$.

$$ p = e^{-\Sigma d} = e^{-\frac{d}{\lambda}} $$

where in this case now $\lambda$ represents the mean free path of the particle in the medium

In simulation, distances are now sampled as

$$ d = -\lambda ln(p) $$

where p is randomly sampled with a uniform distribution between 0 and 1

Inserting this definition for the particle distance into the integrals above
gives

$$ \frac{dp}{dd} = -\frac{p}{\lambda} $$
$$ d = 0 \rightarrow p = 1 $$
$$ d = D \rightarrow p = 0 $$

$$ \int_{0}^{R}\int_{0}^{2\pi}\int_{0}^{\pi}\int_{1}^{0}\int_{0}^{2\pi}\int_{0}^{\pi}
-r^2\sin{\phi} \, \lambda p ln(p)^2 \sin{\alpha} \, \mathrm{d}\alpha \mathrm{d}\beta \mathrm{d}p \, \mathrm{d}\phi
\mathrm{d}\theta \mathrm{d}r $$

This integral can then be evaluated to give the entirety of the geometry space
for the problem

\begin{center}
All Query Space = $\frac{4}{3} \lambda \pi^2 R^3$
\end{center}

To find the portion of utilized space for a given $R$ and $\lambda$ one can
apply the same methods from the fixed distance case, for each particle position
within the sphere there will still be two scenarios - one in which the distance
traveled exceeds $R-d$ and another in which it is less than $R-d$. The difference
now is d has become a distrubution rather than a constanct value, so as the
distance traveled changes the queries will fall into either category for a
single particle position.


Another way of viewing of viewing this condition in the case of a varying
distance is to define these same categories based on the distance traveled. Now
rather than defining how the angular condition is applied based on position, the
angular condition will be applied based on the distance the particle travels.


$$ d < R-r : \alpha_{min} = 0 $$
$$ d > R-r : \alpha_{min} = \arccos\Bigg ( \frac{(2R-r-d)^2-d^2-r^2}{2 d r} \Bigg
) $$

It is interesting to examine plots of the angular limit with varying distance as
seen in Fig. \label{fig:alpha_min}

\begin{figure}[ht] \label{fig:alpha_min}

\centering
\includesvg{alpha_r}{1\textwidth}
\caption{Plot of mininum angle of departure restriction for particles with
various radial positions and varying distances traveled.}
\end{figure}

The entry point of each curve in the plot represents the minimum distance
necessary to enter the $d > R-r$ region and begins to apply an agular
restriction. Note that each of these approach a mininmum angle of $\pi$ as the
distance traveled approaches $R$ (100cm). This represents a particle traveling
back towrd the origin along the sphere's larges chord with length $2R$ and,
again,the maximum distance a particle can travel before the SDF can no longer be
utilized.

Now that the distance traveled is being used to construct these two regions in
the model, this integral must be separated into two pieces, one with limits of $0$
to $R-r$ and another with limits $R-r$ to $R$. Based on the distance sampling
distribution, these values become

$$ d_{min} = R-r \rightarrow p_{max} = e^{(-\frac{(R-r)}{\lambda})} $$
$$ d_{max} = R   \rightarrow p_{min} = e^{(-\frac{R}{\lambda})} $$

and our integral becomes

$$ \int_{0}^{R}\int_{0}^{2\pi}\int_{0}^{\pi}\int_{1}^{p_{max}}\int_{0}^{2\pi}\int_{0}^{\pi}
-r^2sin(\phi) \, \lambda p ln(p)^2 sin(\alpha) \, \mathrm{d}\alpha \mathrm{d}\beta \mathrm{d}p \, \mathrm{d}\phi
\mathrm{d}\theta \mathrm{d}r + $$
$$ \int_{0}^{R}\int_{0}^{2\pi}\int_{0}^{\pi}\int_{p_{max}}^{p_{min}}\int_{0}^{2\pi}\int_{alpha_{min}}^{\pi}
-r^2\sin{\phi} \, \lambda p ln(p)^2 \sin{\alpha} \, \mathrm{d}\alpha \mathrm{d}\beta \mathrm{d}p \, \mathrm{d}\phi
\mathrm{d}\theta \mathrm{d}r $$

The evaluation of this integral divided by the full geometry query space for the
distance sampling scenario after the $c=\frac{R}{\lambda}$ replacement is

$$U(c) = \frac{1}{4}\frac{(4 c^2-6) c^2e^{-2c} - (c^3-c^2-\frac{3}{2}
c+3)4ce^{-2c}+(-2 c^3+2c^2+9c-6) e^{-2c}+4c^2-9 c+6}{c^2}$$

\begin{figure}[ht] \label{fig:sdf_sampled_dist}
\centering
\includesvg{sdf_sampled_dist_results}{\textwidth}
\caption{Results of the model for the theoretical utilization limit with the
results of the simulation for a fixed distance traveled case.}
\end{figure}

Unfortunately the predictive model for the sampled distance scenario is not
consistent with the simulation results, and under predicts the utilization of
the SDF significantly. It is quite possible that utilization in either th interior or exterior
resion of the sphere is being under represented. Regardless of the cause, examining the contribution to
utilization from each of these regions (shown in
Fig. \label{fig:util_region_contributions}) is an interesting endeavor.


\begin{figure}[ht] \label{fig:util_region_contributions}
\centering
\includesvg{../images/util_contributions}{\textwidth}
\caption{A plot of the total predicted utilization along with the contriubtions
from the inner region ($d < R-r$) and outer region ($d > R-r$).}
\end{figure}

When the mean free path $\lambda$ is very small, particles' next event locations
rarely reach the outer region condition. As $\lambda$ increases, particles are
more likely to enter that region and its contribution increases greately. Then
as particles begin to travel distances on the order of the sphere radius the
utilization decreases. The interior region utilization acts as one would
expect. When particles travel small distances with respect to the sphere radius,
there is high utilization, but as the particles begin to travel further its
utilization rapidly decreases. The contribution from the outer region defines the
significance of using both the current position signed distance value as well as the
next event location's signed distance value to precondition ray fire calls in DAGMC.
