
\chapter{Signed Distance Field Model Development and Data}\label{ch:appendix-a}

\section{Detailed Signed Distance Field Model Formulation}

\subsection{Fixed Distance Model Development}

The utilization of the signed distance field as a preconditioner for ray tracing
operations can be modeled as an evaluation of the combined probability space for
particles with a current position, $\vec{p}$, and a next physics event location,
$\vec{n}$, after traveling a distance, $d$. The fraction of this probability
space in which signed distance values can be used to rule out surface crossings
for next surface intersections is then considered to be the theoretical
utilization of the signed distance field. An initial form for this probability
space can found in Equation \ref{appeq:util_model}.

\begin{equation}
  \label{appeq:util_model}
\int_{V_{sphere}}\int_{V_{track}} p_p(r) p_n(d) \, \mathrm{d}V_{sphere}\mathrm{d}V_{track}
\end{equation}

In this model, the starting location of particles, $\vec{p}_{p}(r,\phi,\theta)$, is
uniformly distributed, $p_p(r)=1$, throughout a sphere of radius, $R$, centered
at the origin.  The location of the next event, $\vec{p}_{n}(d,\alpha,\beta)$, where $d$
is the distance traveled by the particle, $\alpha$ is the interior angle between
the particle's \textit{position} vector and the particle's sampled direction
vector, and $\beta$ represents an azimuthal angle for directions traveled with
angle of departure, $\alpha$.  Figure \ref{fig:model} depicts these variables, $r$,
$d$, and $\alpha$ more clearly.

The outer integral in Equation \ref{appeq:util_model} represents all possible particle
positions within the geometric sphere and expands to

\begin{equation}
\int_{0}^{R}\int_{0}^{2\pi}\int_{0}^{\pi}\int_{V_{track}} r^2\sin{\phi} \, \mathrm{d}\phi
\mathrm{d}\theta \mathrm{d}r \,  p_n(d) \mathrm{d}V_{track}
\end{equation}

The inner integral over $V_{track}$ then expands to

\begin{equation}
\small \int_{0}^{R}\int_{0}^{2\pi}\int_{0}^{\pi}\int_{0}^{\infty}\int_{0}^{2\pi}\int_{0}^{\pi}
r^2\sin{\phi} \, p_n(d) d^2 \sin{\alpha} \, \mathrm{d}\alpha \mathrm{d}\beta \mathrm{d}d \, \mathrm{d}\phi
\mathrm{d}\theta \mathrm{d}r
\end{equation}

Integration of $\phi$, $\theta$, and $\beta$ can now be performed with
the knowledge that they are symmetric with respect to the problem and
integration of $p_n(d)$ does not rely on them.

\begin{equation}
\small 8\pi^2  \int_{0}^{R}\int_{0}^{\infty}\int_{0}^{\pi} p_n(d) \,
r^2 \, d^2 \sin{\alpha} \, \mathrm{d}\alpha \mathrm{d}d \, \mathrm{d}r
\end{equation}

In order to represent particles traveling a fixed distance, the relationship in Equation \ref{appeq:pn_fixed}
is applied.

\begin{equation}
  \label{appeq:pn_fixed}
  p_n(d) = \frac{\delta(d-\lambda)}{d^{2}}
\end{equation}

The evaluation of this integral then gives a representation of all the query
space available to the problem

\begin{equation}
  \label{appeq:A_fixed}
\small A = 8\pi^2  \int_{0}^{R}\int_{0}^{\infty}\int_{0}^{\pi} \delta(d-\lambda) \,
r^2 \, \sin{\alpha} \, \mathrm{d}\alpha \mathrm{d}d \, \mathrm{d}r
\end{equation}

and represents all geometric query space, labeled $A$, for a sphere of radius,
$R$ and a fixed distance traveled, $\lambda$. The condition for preconditioner
utilization without error consideration is as follows

\begin{equation}
  SDV(\vec{p}) + SDV(\vec{n}) > |\vec{p}-\vec{n}|
  \label{appeq:condition_no_error}
\end{equation}
\begin{align*}
 &SDV - \, signed \, distance \, value \, function \\
 &\vec{p} - \, particle's \, current \, position \\
 &\vec{n} - \, particle's \, next \, event \, location \\
 &h - \, mesh \, step \, size \\
\end{align*}

To apply this within the spherical geometry, the signed distance function of a
sphere with radius, $R$, from Equation \ref{eq:sdf_sphere} is applied

\begin{equation}
R-|\vec{p}| + R - |\vec{n}| >   |\vec{p}-\vec{n}|
\end{equation}

The right hand side of this inequality can be described as the distance
traveled, $d$, and the magnitude of $\vec{p}$ can be represented
by the variable $r$.

\begin{equation}
 R-r + R - |\vec{n}(d,\alpha,\beta)| > d
\end{equation}

Reducing the next event location, $\vec{n}(d,\alpha,\beta)$, into an expression
in terms of $r$, $d$, and $\alpha$ requires further examination of the
problem. Because the coordinates of $n$ depend on the current particle position,
the magnitude of $n$ with respect to the geometry origin must be obtained to get
a correct form for the signed distance value. Again, Figure \ref{fig:model} depicts the
value of $n$ graphically for reference. The magnitude of n can then be described
using the law of cosines as

\begin{equation}
|n(d,\alpha,\beta)| = \sqrt{r^2 + d^2 - 2rd \cos{\pi-\alpha}}
\end{equation}
inserting this into the inequality gives

\begin{equation}
R-r + R - \sqrt{r^2 + d^2 + 2rd \cos{\alpha}} > d
\end{equation}

The inequality has now been reduced to the three variables seen in Equation
\ref{appeq:A_fixed}: $r$, $d$, and $\alpha$. This can be applied to construct
limits of integration representing boundaries of space in which the SDF can be
utilized. As described in Chapter \ref{ch:preconditioning}, $\alpha_{min}$ can
be used as a limit on the integral over $\textrm{d}\alpha$. It is also mentioned
that $\alpha_{min}$ is undefined until $d > R-r$ as shown in Equations
\ref{appeq:alpha_min_below} and \ref{appeq:alpha_min}.

\begin{equation}
  d < R-r : \alpha_{min} = 0
  \label{appeq:alpha_min_below}
\end{equation}

\begin{equation}
  \alpha_{min} > \arccos\Bigg ( \frac{(2R-r-d)^2-d^2-r^2}{2 d r} \Bigg )
  \label{appeq:alpha_min}
\end{equation}

\begin{figure}[ht]
  \centering
  \includesvg{../images/model_cases_fixed_distance}{width=\textwidth}
  \caption{Depiction of modeling cases. Left: an example of a track for which
    $d < R - r$. Middle: an example of a track for which $R-r < d < R$ and can be
    preconditioned.  Right: an example of a track for which $R-r < d < R$ and
    cannot be preconditioned.}
  \label{appfig:modeling_cases}
\end{figure}

In order to account for the fact that the form of $\alpha_{min}$ is undefined
until $d = R-r$, a Heaviside function is applied before applying it as a limit
on the particle's angle of departure from the position vector. Similarly,
because the $\alpha_{min}$ condition is undefined after $d=R$ a Heaviside
function is used to limit the condition to $\pi$ for any distances traveled
larger than $R$.

\begin{equation}
  \small
  \begin{split}
  \alpha_{min} =& (H(d-(R-r))-H(d-R)) \arccos\Bigg ( \frac{(2R-r-d)^2-d^2-r^2}{2 d r} \Bigg ) \\
  &+ \pi \, H(d-R)
  \end{split}
  \label{appeq:alpha_min_heaviside}
\end{equation}

By inserting this condition as a lower limit of the $d\alpha$ integration,
Equation \ref{appeq:subs_a_cond} will give all utilized space, $US$, in the query space
of the simulation.

\begin{equation}
\small US = 8\pi^2  \int_{0}^{R}\int_{0}^{\infty}\int_{\boldsymbol{\alpha_{min}}}^{\pi} \delta(d-\lambda) \,
r^2 \, d^2 \sin{\alpha} \, \mathrm{d}\alpha \mathrm{d}d \, \mathrm{d}r
\label{appeq:subs_a_cond}
\end{equation}

Evaluating and simplifying this fully formed integral gives us the model
presented in Chapter \ref{ch:preconditioning}.

\begin{equation}
U_{theoretical} = \frac{US}{A} =  \frac{(1-H(\lambda-R))(2R-\lambda)(R-\lambda)}{2R^2}
\end{equation}

\subsection{Sampled Distance Model Development}

%% After the agreement of the simulation results and analytic model for signed
%% distance field utilization for the fixed distance traveled case, the simulation
%% was used to produce a similar set of results in which the distance is
%% sampled based on the standard probability for distance to interaction in a
%% medium with a cross section, $\Sigma$, or mean free path $\lambda
%% =1/\Sigma$. This results in the probability distribution function shown in
%% Equation \ref{appeq:pn_sampled} for the particle distance traveled in this scenario.

%% \begin{equation}
%%   \label{appeq:pn_sampled}
%% p_n(d) \propto \frac{e^{-\Sigma d}}{d^{2}} = \frac{e^{-\frac{d}{\lambda}}}{d^{2}}
%% \end{equation}

%% \begin{figure}[ht]
%% \centering
%% \includesvg{../images/sdf_sampled_dist_results}{width=0.5\textwidth}
%% \caption{Results of the model for the theoretical utilization limit with the
%% results of the simulation for a sampled distance traveled case.}
%% \label{fig:sdf_sampled_dist}
%% \end{figure}

%% %has to be in this section for latex reasons. grumble grumble...
%% \begin{table*}[!h]
%%   \centering
%%   \begin{tabular}{lcccc}
%%           \multicolumn{5}{l}{\textbf{\textit{Source Location:}} <0,0,-1>} \\
%%           \textbf{Implementation} & \textbf{ctme (min)} & \textbf{wall time
%%             (min)} & \textbf{time ratio} & \textbf{precond. utilization}\\
%%           \hline
%%           MCNP6 & 0.17 & 0.14 & 1 & N/A \\
%%           DAG-MCNP6 & 1841.33 & 1841.33 & ~11,000 & N/A \\
%%           DAG-MCNP6 w/ SDF & 0.48 & 0.46 & 2.82 & 0.94\\
%%           \multicolumn{5}{l}{} \\
%%           \multicolumn{5}{l}{\textbf{\textit{Source Location:}} <0,0,10>} \\
%%           \textbf{Implementation} & \textbf{ctme (min)} & \textbf{wall time
%%             (min)} & \textbf{time ratio} & \textbf{precond. utilization}\\
%%           \hline
%%           MCNP6 & 0.18 & 0.18 & 1 & N/A \\
%%           DAG-MCNP6 & 11.12 & 11.16 & 62 & N/A \\
%%           DAG-MCNP6 w/ SDF & 0.50 & 0.52 & 2.89 & 0.96 \\
          
%%   \end{tabular}
%%   \caption{Performance results for an MCNP6 test case involving electron
%%     transport of a 1 keV-100 keV photon source incident on an Fe/W target. 5,000
%%     histories were run in this test problem.}
%%   \label{tab:inp066_results}
%% \end{table*}

%% Following the same process as in the fixed distance case by plugging Equation \ref{appeq:pn_sampled} into
%% Equation \ref{appeq:subs_a_cond}, the utilization form for the sampled distance case is
%% shown in Equation \ref{appeq:sampled_limit}.

%% \begin{equation}
%%   \label{appeq:sampled_limit}
%%   U_{theoretical} = \frac{US}{A} = \frac{ \frac{1}{2} \lambda(R - 2 \lambda) e^{\frac{-R}{\lambda}} + \lambda^2 - \frac{3}{2} R \lambda + R^2 }{R^2}
%% \end{equation}

%% The results of this set of simulations can be seen in
%%  Figure \ref{fig:sdf_sampled_dist}. In this scenario, it is not expected that the
%% utilization will approach zero when $\lambda = 100\, cm$, as the actual distance
%% sampled may be considerably less than the provided mean free path for the
%% simulation. Overall utilization values in this scenario for $\lambda$ from 0 to
%% 100 cm remain higher than the corresponding fixed distance simulation cases as
%% is expected in a sampled distance case. Utilization values remain high for
%% relatively large increases in mesh step size, $h$. This is important to
%% application of the data structure given concerns regarding its potentially high
%% memory footprint for large volumes. For example if the utilization of the signed
%% distance field drops $~20\%$ when going from a step size of 1 cm to 6.21 cm, but
%% the memory footprint of the data structure will have decreased by a factor of
%% $6.21^3$ or $239.5$ as well. The optimization of the mesh step size with respect
%% to its effect on utilization will also need to be included in future models of
%% the utilization.


The sampled distance probability distribution presented in Chapter \ref{ch:preconditioning}
is as follows:

\begin{equation}
  p = \frac{e^{-\Sigma d}}{d^{2}} = \frac{e^{-\frac{d}{\lambda}}}{d^{2}}
\end{equation}

with distances sampled using the function

\begin{equation}
  d = -\lambda ln(c)
\end{equation}

where $c$ is randomly sampled with a uniform distribution between 0 and 1

An examination of the change of variables in the general form for the utilized
space from Equation \ref{eq:util_model} gives

\begin{equation}
  \frac{dp}{dd} = -\frac{p}{\lambda}
\end{equation}

\begin{equation}
  d = 0 \rightarrow c = 1
\end{equation}

\begin{equation}
  d = \infty \rightarrow c = 0
\end{equation}

Resulting in an integral with the following form for the sampled distance model

\begin{equation}
  \int_{0}^{R}\int_{0}^{2\pi}\int_{0}^{\pi}\int_{1}^{0}\int_{0}^{2\pi}\int_{0}^{\pi}
-r^2\sin{\phi} \, \lambda c ln(c)^2 \sin{\alpha} \, \mathrm{d}\alpha \mathrm{d}\beta \mathrm{d}c \, \mathrm{d}\phi
\mathrm{d}\theta \mathrm{d}r
\end{equation}

As in the fixed distance case, the condition for $\alpha_{min}$ is a piece-wise function based on the distance traveled. The expression for d changes slightly for this case however, given that the integral is now being performed over the variable $c$.

\begin{equation}
  d < R-r : \alpha_{min} = 0
\end{equation}

\begin{equation}
  d > R-r : \alpha_{min} = \arccos\Bigg ( \frac{(2R-r-d)^2-d^2-r^2}{2 d r} \Bigg
  )
\end{equation}

Now that the distance traveled is being used to construct these two regions in
the model, this integral must be separated into two pieces, one with limits of $0$
to $R-r$ and another with limits $R-r$ to $R$. Based on the distance sampling
distribution, these values become

\begin{equation}
  d_{min} = R-r \rightarrow c_{max} = e^{(-\frac{(R-r)}{\lambda})}
\end{equation}

\begin{equation}
  d_{max} = R   \rightarrow c_{min} = e^{(-\frac{R}{\lambda})}
\end{equation}

and the resulting integral becomes

\begin{equation}
  \int_{0}^{R}\int_{0}^{2\pi}\int_{0}^{\pi}\int_{1}^{c_{max}}\int_{0}^{2\pi}\int_{0}^{\pi}
-r^2sin(\phi) \, \lambda c ln(c)^2 sin(\alpha) \, \mathrm{d}\alpha \mathrm{d}\beta \mathrm{d}c \, \mathrm{d}\phi
\mathrm{d}\theta \mathrm{d}r +
\end{equation}

\begin{equation}
  \int_{0}^{R}\int_{0}^{2\pi}\int_{0}^{\pi}\int_{c_{max}}^{c_{min}}\int_{0}^{2\pi}\int_{alpha_{min}}^{\pi}
-r^2\sin{\phi} \, \lambda c ln(c)^2 \sin{\alpha} \, \mathrm{d}\alpha \mathrm{d}\beta \mathrm{d}c \, \mathrm{d}\phi
\mathrm{d}\theta \mathrm{d}r
\end{equation}

The evaluation of this integral gives the final form of the analytic preconditioner limit from Equation \ref{eq:sdf_model_sampled}

\begin{align}
    \begin{split}
    U_{sampled} = \frac{\frac{1}{2} \lambda (R - 2 \lambda) e^{-\frac{R}{\lambda}} + \lambda^{2} - \frac{3}{2} R \lambda + R^{2} }{R^{2}}
  \end{split}
\end{align}

%% It is quite possible that utilization in either th interior or exterior
%% resion of the sphere is being under represented. Regardless of the cause, examining the contribution to
%% utilization from each of these regions (shown in
%%  Figure \label{fig:util_region_contributions}) is an interesting endeavor.


%% \begin{figure}[ht] 
%% \centering
%% \includesvg{../images/util_contributions}{width=\textwidth}
%% \caption{A plot of the total predicted utilization along with the contriubtions
%%   from the inner region ($d < R-r$) and outer region ($d > R-r$).}
%% \label{fig:util_region_contributions}
%% \end{figure}

%% Figure \ref{fig:util_region_contributions} depicts the utilization of the SDF
%% data structure for various values of the mean free path, $\lambda$.
%% When $\lambda$ is very small, particles' next event locations
%% rarely reach the outer region condition of $ d > R - r $. As $\lambda$ increases, particles are
%% more likely to enter that region and its contribution increases greatly. Then
%% as particles begin to travel distances on the order of the sphere radius the
%% utilization decreases. The interior region utilization acts as one would
%% expect. When particles travel small distances with respect to the sphere radius,
%% there is high utilization, but as the particles begin to travel further its
%% utilization rapidly decreases. The contribution from the outer region defines the
%% significance of using both the current position signed distance value as well as the
%% next event location's signed distance value to precondition ray fire calls in DAGMC.

\subsection{Inclusion of Error in Model Development}


As provided in Chapter \ref{ch:preconditioning}, the condition for avoiding a
ray fire call when including error associated with signed distance value
interpolation is:

\begin{equation}
  SDV(\vec{p}) + SDV(\vec{n}) > |\vec{p}-\vec{n}| + 2\varepsilon(h)
  \label{appeq:condition}
\end{equation}
\begin{align*}
 &SDV - \, signed \, distance \, value \, function \\
 &\vec{p} - \, particle's \, current \, position \\
 &\vec{n} - \, particle's \, next \, event \, location \\
 &h - \, mesh \, step \, size \\
 &\varepsilon(h) - \, error \, evaluation \, for \, signed \, distance \, values \\
\end{align*}

The limits of the $\alpha_{min}$ condition need to be adjusted yet again to
account for the error that will be subtracted from the signed distance values. This becomes a relatively straightforward process in which the error is also subtracted from the arguments to the Heaviside functions in Equation \ref{appeq:alpha_min_heaviside}. Accounting for interpolation error reduces the maximum possible distance the particle can travel and still be preconditioned. It also reduces the value of $d_{min}$ where $\alpha_{min}$ becomes non-zero.

\begin{equation}
\small
\begin{split}
\alpha_{min} =& \arccos\Bigg ( \frac{(2R-r-d-2\epsilon)^2-d^2-r^2}{2 d r} \Bigg )[H(d-(R-r-\epsilon))-H(d-(R-\epsilon))]  \\
&+ \pi \, H(d-(R-\epsilon))
\end{split}
\label{appeq:min_alpha_w_error}
\end{equation}

After these adjustments to the $\alpha_{min}$ condition, the integral can be evaluated and simplified to give the form found in Equation \ref{eq:sdf_util_sampled_w_error}:

\begin{equation}
  US_{sampled} = \frac{(R-\epsilon) (\frac{1}{2} \lambda ( R - 2\lambda - \epsilon ) e^{\frac{-R + \epsilon}{\lambda}} + \lambda^{2} - \frac{3}{2}\lambda(R - \epsilon) + (R-\epsilon)^{2})}{R^3}
  \label{appeq:sdf_util_sampled_w_error}
\end{equation}

%% For a direct evaluation of the utilization based on
%% the mesh step size, $h$, the formula for the error can be substituted for
%% $\epsilon$ to give

%% \begin{equation}
%% \small
%% \begin{split}
%% \alpha_{min} =& (H(d-(R-r-\sqrt{3}h))-H(d-(R-\sqrt{3}h))) \arccos\Bigg ( \frac{(2R-r-d-2\sqrt{3}h)^2-d^2-r^2}{2 d r} \Bigg ) \\
%% &+ \pi \, H(d-(R-\sqrt{3}h))
%% \end{split}
%% \end{equation}

