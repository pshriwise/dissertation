
\chapter{Geometry Query Preconditioning}\label{ch:preconditioning}

This chapter describes the adaptation of a rendering data structure, the signed
distance field, as a geometric query tool for accelerating CAD-based transport
in the Direct Accelerated Monte Carlo (DAGMC) toolkit. The beginnings of a
predictive model for the data structure's utilization based on various problem
parameters is also introduced and demonstrations of its effectiveness are shown
for a number of simple problems as well two production level models.

\section{Signed Distance Field Preconditioning}\label{section:preconditioner_theory}

Three types of geometric queries are common among the various Monte Carlo codes
that DAGMC supports. These include next surface intersection, point containment,
and closest boundary intersection queries. Typically, a ray is fired to satisfy
any of these queries in DAGMC with $O(logN)$ complexity, but it is hypothesized
that these queries can be accelerated in many cases by first performing an
$O(1)$ signed distance value look-up to precondition the higher complexity ray
fire calls and make sure they are necessary.

Point containment queries can be performed by examining the signed distance
value for the current particle location. If the point's value is negative (or
outside of the signed distance field data structure), then the point is not
contained within the volume. If the point's value is positive, then the point is
determined to be inside the volume. Given that there is error associated with
each of these interpolated values, the point containment using a signed distance
field should only be trusted if the absolute value of the signed distance is
greater than the expected error associated with the value. If this is not the
case, then a ray must be fired to determine the particle's containment with
respect to the volume in question.

Closest to boundary queries can be performed in a similar manner to the point
containment queries, but they are more dependent on the native code's intent for
their use. Some Monte Carlo codes query for the nearest volume surface
intersection in order to determine whether or not the particle will exit the
volume before reaching its next physics event location. Signed distance fields are
designed for exactly this operation. In similar fashion to the point
containment case, the signed distance value should only be trusted if it is greater than the
error associated with the value. Additionally, the error should be subtracted
from the value, returning to the code a conservative value for the nearest
intersection. If the signed distance value's magnitude is not greater than its
error evaluation or if the value is negative, then a ray should be fired to
determine the exact location of the nearest boundary crossing for the particle's
location.

Next surface intersections are called by native Monte Carlo codes to determine
if a particle will cross a surface before reaching its next physics event
location. This is the most common geometry query in an average Monte Carlo
simulation. Normally in DAGMC a ray is fired each time this query is
called. This can be avoided by using the signed distance field to exclude the
possibility of a surface crossing without explicitly determining the next
surface intersection. If the sum of the signed distance values for both the
current particle position and the next physics event location is greater than
the distance between the two, then no surface crossing will occur and the
particle can safely advance to the next physics event location. For robustness,
the error for each interpolation should be subtracted from the sum of the signed
distance values as a conservative measure. If the expanse between the particle's
current location and its next physics event interaction cannot be accounted for
by the signed distance values of the two points, then a ray will be fired to
determine the particle's next surface intersection along that
trajectory. Fig. \ref{fig:precondition_ray} shows a graphic representation of
this process. Not all Monte Carlo codes provide the next physics event location
along with the particle's current location to their geometry kernels. In this
case, preconditioning of these queries will not be possible.

\begin{figure}[ht]
  \center
  \includesvg{../images/preconditioner_ex}{0.35\textwidth}
  \caption{Visualization of ray preconditioning scenarios. The $\epsilon$ here represents
    the associated error of the signed distance value interpolation.}
  \label{fig:precondition_ray}
\end{figure}

Using these methods, the storage of signed distances field for volumes in DAGMC
could provide a way to accelerate the Monte Carlo queries listed above by using
a $O(1)$ process to establish that the conditions of a geometric query are such
that a more robust and computationally expensive ray is necessary before
performing the ray tracing operation. It is hypothesized that in many cases
during radiation transport that this process can be used to subvert many ray
tracing calls and a large number of these $O(logN)$ searches can be avoided.
