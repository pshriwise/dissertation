%% \textbf{FIXME:  basically a placeholder; do not believe}

%% \svnidlong{$LastChangedBy$}{$LastChangedRevision$}{$LastChangedDate$}{$HeadURL: http://freevariable.com/dissertation/branches/diss-template/frontmatter/abstract.tex $}
%% \vcinfo{}

%% I did some research, read a bunch of papers, published a couple myself, (pick one):
%% \begin{enumerate}
%% 	\item ran some experiments and made some graphs,
%% 	\item proved some theorems
%% \end{enumerate}
%% and now I have a job.  I've assembled this document in the last couple of months so you will let me leave.  Thanks!

The performance of CAD-based Monte Carlo Radiation Transport (MCRT) relies
heavily on its ability to return geometric queries robustly via ray
tracing methods. Current applications of ray tracing for MCRT are robust given that certain
requirements are met \cite{Smith_2011}, but cause simulations to run much longer
than native code geometry representations. This work explores alternate geometry
query methods aimed at reducing the complexity of these operations as well as
algorithmic optimization by adapting recent developments in CPU ray
tracing for use in engineering analysis. A preconditioning scheme is presented
aimed at avoiding unnecessary ray queries for volumes with high collision
densities. A model is also developed to inform the application of the
preconditioning data structure based on a \textit{post facto} analysis. Next, a
specialized ray tracing kernel for MCRT is presented. As new ray tracing kernels
are developed for real-time, photo-realistic rendering, algorithmic approaches
have appeared which are demonstrated to be advantageous when applied in radiation transport. In
particular, the application of data parallelism in ray tracing for Monte Carlo
is demonstrated - resulting in significant performance improvements. Finally,
model features resulting in systematic performance degradation commonly found in
CAD models for MCRT are studied. Methods are proposed and demonstrated to
improve performance of ray tracing kernels in models with these features. The combination
of this work is shown to provide improvement factors ranging from 1.1 to 9.5 in simulation run time
without loss of robustness for several production analysis models. The final impact
of this work is the alleviation of concern for additional computational time in
using CAD geometries for MCRT while maintaining the benefit of reduced human
time and effort in model preparation and design.
