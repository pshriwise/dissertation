%% \begin{wbepi}{David C.~Makinson (1965)}
%% It is customary for authors of academic books to include in their prefaces statements such as this: ``I am indebted to ... for their invaluable help; however, any errors which remain are my sole responsibility.'' Occasionally an author will go further. Rather than say that if there are any mistakes then he is responsible for them, he will say that there will inevitably be some mistakes and he is responsible for them....

%% Although the shouldering of all responsibility is usually a social ritual, the admission that errors exist is not --- it is often a sincere avowal of belief. But this appears to present a living and everyday example of a situation which philosophers have commonly dismissed as absurd; that it is sometimes rational to hold logically incompatible beliefs.
%% \end{wbepi}

%% Above is the famous ``preface paradox,'' which illustrates how to use
%% the \texttt{wbepi} environment for epigraphs at the beginning of
%% chapters.  You probably also want to thank the Academy.


I'm thankful to a number of colleagues, teammates, friends, and family members who have supported me
throughout the course of my PhD work. In particular, I'd like to thank my
advisor, Dr. Paul Wilson, who has been an excellent mentor and friend to me. His
guidance and wisdom has been critical in my development as a researcher.

To my colleagues in the Computational Nuclear Engineering Research Group
(CNERG), thank you for your support and friendship. 

I thank other mentors of mine as well. I thank Dr. Andrew for the long
discussions in his office and professional guidance. I'm grateful to Dr. Tim Tautges as
well for his wealth of knowledge in the meshing field, and for enabling my
caffiene addiction by supplying regular espresso breaks.

This work was funded in part by the Nuclear Regulatory Commission.
