%% \begin{wbepi}{David C.~Makinson (1965)}
%% It is customary for authors of academic books to include in their prefaces statements such as this: ``I am indebted to ... for their invaluable help; however, any errors which remain are my sole responsibility.'' Occasionally an author will go further. Rather than say that if there are any mistakes then he is responsible for them, he will say that there will inevitably be some mistakes and he is responsible for them....

%% Although the shouldering of all responsibility is usually a social ritual, the admission that errors exist is not --- it is often a sincere avowal of belief. But this appears to present a living and everyday example of a situation which philosophers have commonly dismissed as absurd; that it is sometimes rational to hold logically incompatible beliefs.
%% \end{wbepi}

%% Above is the famous ``preface paradox,'' which illustrates how to use
%% the \texttt{wbepi} environment for epigraphs at the beginning of
%% chapters.  You probably also want to thank the Academy.


I'm thankful to a number of colleagues, teammates, friends, and family members
who have supported me throughout the course of my PhD work. To begin, I'd like
to thank some of the professional mentors I feel very lucky to have worked with.

To Dr. Paul Wilson goes my extreme gratitude for taking me on as a student in the
Computational Engineering Research Group (CNERG). He is a shining example of
integrity, wisdom, and intelligence for his 
students. These qualities along with a dedicated work ethic are reflected in all of
the work that comes out of our research. His holistic view of the graduate
student experience and ability to develop an inclusive culture in his group is
something of great value to his students, whether they 
know it or not. I couldn't have asked for a better advisor.

To Dr. Tim Tautges, thank you for all of the conversations in your office about
life, politics, research, and otherwise. I'll cherish those times greatly. Your
wealth and breadth of knowledge in the field and academic rigor have been of
immense value to me as a researcher, and your friendship even more
so. Additionally, thank you for providing me with the caffeine necessary to
survive the long weeks of writing and research.

I'd like to thank Dr. Andrew Davis for being a daily source of support and
positivity in his time at UW. His uplifting attitude and generally helpful
nature are missed here on the fourth floor. The hours you've saved students through
your willingness to aide and impart knowledge cannot be counted. I thank Andy
also for rather enjoyable discussions on ideas about somewhat extreme research
concepts and his knowledge about the landscapes of computational nuclear
engineering.

My thanks to all of my colleagues in CNERG who are not only coworkers but
friends to me. In particular, thank you to Kalin for a variety of entertaining
discussions along with insight into what it is to be a woman in STEM
today. Lucas, thank you for all of the hard work you've put in maintaining and
extending DAGMC. Alex and Baptiste, thank you for making the fourth floor such
an enjoyable place to work, full of laughter and fun.

My heartfelt thanks goes out to all of my ultimate frisbee teammates and
coaches. This sport has been a critical source of stress relief, grounding and
perspective for me throughout my time in Madison. I've met so many wonderful
people across the globe through it, making life-changing and life-long
friendships in the process.

To my parents, Rod and Susan, thank you for all of your love and support. You've
ingrained in me a set of principles, values, and work ethic which have served me
will for many years and many more to come. Amanda, my sister, thank you for
being a strong academic role model to me and for helping me to bet on myself
more often. My gratitude as well to Marcia Bosscher, my mother-in-law, for many delicious
meals, hosted holidays, and general support. Finally, I'd like to thank my wife,
Georgia. You are a constant source of support and joy in my life. Your
dedication to your profession of veterinary surgery drives me to show the same
to mine. I will forever look back fondly on the time we've spent walking
our dog, Monk, along the water at the edge of the arboretum sharing our hopes
and dreams, our fears and frustrations, with each other.

This work was funded in part by the Nuclear Regulatory
Commission. My thanks for their financial support and all they do for the
nuclear engineering community.
